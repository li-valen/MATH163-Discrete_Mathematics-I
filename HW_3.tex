\documentclass[12pt]{article}
\usepackage{latexsym, color, amssymb}
\voffset=-.8cm
\hoffset=-1.5cm
\setlength{\textheight}{23cm}
\setlength{\textwidth}{15cm}
\pagestyle{myheadings}
\newtheorem{q}{Problem}
\newtheorem{dfntn}{Definition}
\newcommand{\df}{\displaystyle\frac}
\newcommand{\beq}{\begin{q} \hskip-.3cm)}
\newcommand{\eeq}{\end{q}\newpage}
\newcommand{\ra}{\rightarrow}
\newcommand{\rla}{\leftrightarrow}
\markright{Dr. Petrescu CCP MATH163 Homework 3 - Solutions}
\begin{document}
\vskip.1cm{\bf Show all your work to get credit.} \vskip0.3cm
{\bf Name}: Valen Li\ \vskip.5cm
{\bf Date}: September 21, 2025 \vskip.5cm

%%%% Problem 1
\begin{q}
Prove that \(\neg [r \vee (q \wedge (\neg r \rightarrow \neg p))] \equiv \neg r \wedge (p \vee \neg q)\) by using a series of logical equivalences.
\end{q}
{\bf Solution:} 

\begin{itemize}
    \item Simplify the implication: \(\neg r \ra \neg p \equiv \neg (\neg r) \vee \neg p \equiv r \vee \neg p\). Thus, the expression becomes:
    \[
    \neg [r \vee (q \wedge (r \vee \neg p))].
    \]
    \item Apply De Morgan’s Law to the negation of the disjunction:
    \[
    \neg [r \vee (q \wedge (r \vee \neg p))] = \neg r \wedge \neg (q \wedge (r \vee \neg p)).
    \]
    \item Apply De Morgan’s to the negated conjunction:
    \[
    \neg (q \wedge (r \vee \neg p)) = \neg q \vee \neg (r \vee \neg p).
    \]
    So the expression is:
    \[
    \neg r \wedge (\neg q \vee \neg (r \vee \neg p)).
    \]
    \item Simplify \(\neg (r \vee \neg p)\):
    \[
    \neg (r \vee \neg p) = \neg r \wedge \neg (\neg p) = \neg r \wedge p.
    \]
    Thus:
    \[
    \neg r \wedge (\neg q \vee (\neg r \wedge p)).
    \]
    \item Distribute \(\neg q \vee (\neg r \wedge p)\):
    \[
    \neg q \vee (\neg r \wedge p) = (\neg q \vee \neg r) \wedge (\neg q \vee p).
    \]
    So the expression becomes:
    \[
    \neg r \wedge (\neg q \vee \neg r) \wedge (\neg q \vee p).
    \]
    \item Apply the absorption law:
    \[
    \neg r \wedge (\neg q \vee \neg r) = \neg r.
    \]
    Resulting in: 
    \[
    \neg r \wedge (\neg q \vee p).
    \]
    \item Recognize that \(\neg q \vee p = p \vee \neg q\) (commutative property), so:
    \[
    \neg r \wedge (p \vee \neg q).
    \]
    This matches the right hand side we were given at the start.
\end{itemize}

\eeq

%%%% Problem 2
\begin{q}
Express the following propositions using quantifiers, then express the negation in English and using quantifiers: \\
(a) Some people have no common sense. \\
(b) All Swedish movies are boring. \\
(c) No one can keep a secret. \\
(d) Someone in this class has a bad attitude. \\
Make sure you indicate the predicate and its domain.
\end{q}
{\bf Solution:}

\begin{itemize}
    \item[(a)] {\bf Proposition:} Some people have no common sense.
    \begin{itemize}
        \item {\bf Predicate:} \(C(x) = \) ``x has common sense.''
        \item {\bf Domain:} All people.
        \item {\bf Quantified Form:} \(\exists x \neg C(x)\).
        \item {\bf Negation (English):} ``All people have common sense.''
        \item {\bf Negation (Quantifiers):} \(\neg (\exists x \neg C(x)) \equiv \forall x C(x)\).
    \end{itemize}

    \item[(b)] {\bf Proposition:} All Swedish movies are boring.
    \begin{itemize}
        \item {\bf Predicate:} \(B(x) = \) ``x is boring.''
        \item {\bf Domain:} All Swedish movies.
        \item {\bf Quantified Form:} \(\forall x B(x)\).
        \item {\bf Negation (English):} ``Some Swedish movies are not boring.''
        \item {\bf Negation (Quantifiers):} \(\neg (\forall x B(x)) \equiv \exists x \neg B(x)\).
    \end{itemize}

    \item[(c)] {\bf Proposition:} No one can keep a secret.
    \begin{itemize}
        \item {\bf Predicate:} \(K(x) = \) ``x can keep a secret.''
        \item {\bf Domain:} All people.
        \item {\bf Quantified Form:} \(\forall x \neg K(x)\).
        \item {\bf Negation (English):} ``Someone can keep a secret.''
        \item {\bf Negation (Quantifiers):} \(\neg (\forall x \neg K(x)) \equiv \exists x K(x)\).
    \end{itemize}

    \item[(d)] {\bf Proposition:} Someone in this class has a bad attitude.
    \begin{itemize}
        \item {\bf Predicate:} \(A(x) = \) ``x has a bad attitude.''
        \item {\bf Domain:} All people in this class.
        \item {\bf Quantified Form:} \(\exists x A(x)\).
        \item {\bf Negation (English):} ``No one in this class has a bad attitude.''
        \item {\bf Negation (Quantifiers):} \(\neg (\exists x A(x)) \equiv \forall x \neg A(x)\).
    \end{itemize}
\end{itemize}

\eeq

%%%% Problem 3
\begin{q}
Let \(M(x) = \) ``x is a millionaire'' and \(P(x) = \) ``x drives a Porsche''. The domain is all people. Translate to English: \\
(a) \(\forall x (M(x) \ra P(x))\) \\
(b) \(\exists x (M(x) \ra P(x))\) \\
(c) \(\forall x (M(x) \wedge P(x))\) \\
(d) \(\exists x (M(x) \vee P(x))\)
\end{q}
{\bf Solution:}

\begin{itemize}
    \item[(a)] \(\forall x (M(x) \ra P(x))\): ``All millionaires drive a Porsche.''
    \item[(b)] \(\exists x (M(x) \ra P(x))\): Since \(M(x) \ra P(x) \equiv \neg M(x) \vee P(x)\), this translates to: ``There exists a person who is either not a millionaire or drives a Porsche.''
    \item[(c)] \(\forall x (M(x) \wedge P(x))\): ``Everyone is both a millionaire and drives a Porsche.''
    \item[(d)] \(\exists x (M(x) \vee P(x))\): ``There exists a person who is either a millionaire or drives a Porsche.''
\end{itemize}

\eeq

%%%% Problem 4
\begin{q}
Prove that \(\cos x\) and \(\sin x\) are continuous \(\forall x \in \mathbb{R}\).
\end{q}
(I didn't understand how to prove this, so I watched a YouTube video that explained it and I only kinda get it so heres my attempt at the proof) \\ \\
{\bf Solution:} 



To prove that \(\sin x\) is continuous at every point \(a \in \mathbb{R}\), we need to show that for every \(\epsilon > 0\), there exists a \(\delta > 0\) such that if \(|x - a| < \delta\), then \(|\sin x - \sin a| < \epsilon\).

Start with the trigonometric identity:
\[
\sin x - \sin a = 2 \cos\left(\frac{x + a}{2}\right) \sin\left(\frac{x - a}{2}\right).
\]
Taking the absolute value:
\[
|\sin x - \sin a| = 2 \left|\cos\left(\frac{x + a}{2}\right)\right| \left|\sin\left(\frac{x - a}{2}\right)\right|.
\]
Since \(\left|\cos \theta\right| \leq 1\) for any \(\theta\), this simplifies to:
\[
|\sin x - \sin a| \leq 2 \left|\sin\left(\frac{x - a}{2}\right)\right|.
\]
We use the known inequality \(|\sin \theta| \leq |\theta|\) for all real \(\theta\) (this can be established geometrically: in the unit circle, the vertical distance \(\sin \theta\) is less than or equal to the arc length \(\theta\) for \(\theta \geq 0\), and by symmetry for \(\theta < 0\)):
\[
\left|\sin\left(\frac{x - a}{2}\right)\right| \leq \left|\frac{x - a}{2}\right| = \frac{|x - a|}{2}.
\]
Substituting back:
\[
|\sin x - \sin a| \leq 2 \cdot \frac{|x - a|}{2} = |x - a|.
\]
To make \(|\sin x - \sin a| < \epsilon\), choose \(\delta = \epsilon\). Then, if \(|x - a| < \delta = \epsilon\), we have \(|\sin x - \sin a| \leq |x - a| < \epsilon\).

This \(\delta\) works for any \(a\), so \(\sin x\) is continuous everywhere.
\\ \\
Similarly, for \(\cos x\) at \(a \in \mathbb{R}\):
\[
\cos x - \cos a = -2 \sin\left(\frac{x + a}{2}\right) \sin\left(\frac{x - a}{2}\right).
\]
Taking the absolute value:
\[
|\cos x - \cos a| = 2 \left|\sin\left(\frac{x + a}{2}\right)\right| \left|\sin\left(\frac{x - a}{2}\right)\right|.
\]
Since \(\left|\sin \theta\right| \leq 1\) for any \(\theta\):
\[
|\cos x - \cos a| \leq 2 \cdot 1 \cdot \left|\sin\left(\frac{x - a}{2}\right)\right| = 2 \left|\sin\left(\frac{x - a}{2}\right)\right|.
\]
Again, using \(|\sin \theta| \leq |\theta|\):
\[
2 \left|\sin\left(\frac{x - a}{2}\right)\right| \leq 2 \cdot \frac{|x - a|}{2} = |x - a|.
\]
So:
\[
|\cos x - \cos a| \leq |x - a|.
\]
Choose \(\delta = \epsilon\). If \(|x - a| < \delta = \epsilon\), then \(|\cos x - \cos a| < \epsilon\).

This works for any \(a\), so \(\cos x\) is continuous everywhere.
\eeq

\end{document}